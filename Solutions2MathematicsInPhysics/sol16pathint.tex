% !TEX program = xelatex


\documentclass[12pt,a4paper]{article}


%%%%% Page settings  %%%%%
\addtolength{\textheight}{2.0cm}
\addtolength{\voffset}{-2cm}
\addtolength{\hoffset}{-1.0cm}
\addtolength{\textwidth}{2.0cm}

%\allowdisplaybreaks


%%%%%% Math %%%%%%%
\usepackage{amsthm,amsfonts,amssymb,bm}
\usepackage{mathrsfs}
\usepackage[fleqn]{amsmath}
\usepackage{subeqnarray}




%%%%% MISC  %%%%%%

%\usepackage{color}
\usepackage[usenames,dvipsnames]{color}
\usepackage{url}
\usepackage{ulem}
\usepackage{indentfirst}   % Indent first line of a paragraph
%\usepackage{textcomp}

\usepackage{enumerate}


%%%%%%   Here is the configuration for chinese. setmainfont is the default font of the text.
%\usepackage[cm-default]{fontspec}
%\usepackage{xunicode}
%\usepackage{xltxtra}
%\setmainfont{Arial}
%\setsansfont[BoldFont=Arial]{KaiTi_GB2312}
%\setmonofont{NSimSun}


%\XeTeXlinebreaklocale "zh"
%\XeTeXlinebreakskip = 0pt plus 1pt

% Redefine some fonts.
%\newfontfamily\heiti{"黑体"}
%\newfontfamily\fs{"仿宋"}
%\newfontfamily\yahei{"微软雅黑"}


%%%%%%% Figure, Diagram, Caption settings  %%%%%
%\usepackage{tikz}
%\usetikzlibrary{mindmap,trees}

\usepackage{graphicx}
%\usepackage{graphics}
%\usepackage[hang,small,bf]{caption}
%\setlength{\captionmargin}{50pt}


%\graphicspath{{Figures/}



%includeonly{}
\usepackage{comment}

\begin{document}
%\title{TITLE}
%\author{{\bf MA} Lei  \\
%@ Interplanetary Immigration Agency \\
%{\small\em \copyright \ Draft date \today}}
%\date{}
%\maketitle


%%%%%% Redefine some math commands and environments. %%%%

%\newcommand{\dd}{ d }

%\newcommand{\HH}{\mathcal H}

%\newcommand{\CN}{{\it Cosmologia Notebook}}

%\newenvironment{eqnset}
%{\begin{equation}\left \bracevert \begin{array}{l}}
%{\end{array} \right. \end{equation}}

%\newenvironment{eqn}
%{\begin{equation}\left \bracevert \begin{array}{l}}
%{\end{array} \right. \end{equation}}


%%%%%%%%%%%%%%%%%%%%%%%%%%%%%%%%%%%%%%%%%%%%%%%%%%%%%%%%
%%%%%%%%%%%%%%    Let's Start Typing     %%%%%%%%%%%%%%%
%%%%%%%%%%%%%%%%%%%%%%%%%%%%%%%%%%%%%%%%%%%%%%%%%%%%%%%%





%%%%%%%%  Tensors and Local Symmetries   %%%%%%%%%%
\section{Path Integral}

\begin{enumerate}


%\begin{comment}




%%%%%%%%%%%   1 %%%%%%%
%%16.1 Derive the multiple gaussian integral (17.8) from (5.178). %%%%

\item
\begin{eqnarray*}
	&&\int_{-\infty}^\infty \exp\left(\sum_i - r_i x_i^2 + c_i x_i\right) \prod_{i=1}^N  d x_i \\
	&=& \prod_i \int_{-\infty}^\infty \exp\left( -r_ix_i^2 + c_i x_i \right) d x_i \\
	&=& \int_{-\infty}^{\infty} \prod_i \left( \exp\left( -r_i (x_i - \frac{c_i}{2r_i})^2 \right)\exp\left( \frac{c_i^2}{4r_i} \right) d x_i \right) \\
	&=& \prod _{i=1}^N \sqrt{\frac{\pi}{r_i}} \exp\left( \frac14 \sum_i \frac{c_i^2}{r_i} \right)
\end{eqnarray*}





%%%%%   16.2  %%%%%
%%%%   Derive the multiple gaussian integral (17.12) from (5.177).
\item

The matrix form of
\begin{eqnarray*}
	&&\int_{-\infty}^{\infty} \exp\left( \sum_i ( -i a_i x_i^2  + i b_i x_i  )\prod _i  d  x_i \right) \\
	&=& \prod_{i=1}^N \sqrt{\frac{\pi}{ia_i}}\exp\left( \frac{i}{4} \sum_i \frac{b_i^2}{a_i} \right)
\end{eqnarray*}
is
\begin{eqnarray*}
	\int_{-\infty}^{\infty} \exp\left( -i X^T A X + i B X \right) \prod_{i=1}^N  d x_i &=& \sqrt{\frac{\pi^N}{\det(iA)}}  \exp\left( \frac{i}{4} B^T A^{-1}B \right)
\end{eqnarray*}

We know that $A = O^T S O$,
\begin{eqnarray*}
	\int_{-\infty}^\infty \exp\left( -i X^T O^T S O X + i B O^T O X \right)\prod_{i=1}^N  d x_i &=& \sqrt{\frac{\pi^N}{\det(S)}} \exp\left( \frac i 4  B^T O^T O A^{-1} O^T O B \right)
\end{eqnarray*}

Since $Y=OX$, $D=OB$,
\begin{eqnarray*}
	\int_{-\infty}^\infty \exp\left( -i Y^T S Y +  i D^T Y \right)\prod_{i=1}^N  d y_i &=& \sqrt{\frac{\pi^N}{\det(S)}} \exp\left( \frac i 4 D^T S^{-1} D \right) 
\end{eqnarray*}





%%%%%%%  16.3 %%%%
%%%  Show that the vector Y that makes the argument of the multiple gaussian integral (17.12) stationary is given by (17.13), and that the multiple gaussian integral (17.12) is equal to its exponential evaluated at its stationary point Y apart from a prefactor involving the determinant det iS.  %%%%%%%%
\item

Denote
\begin{equation*}
	I = \int _{-\infty}^\infty  \exp\left( -iY^T S Y + i D^T Y \right) \prod_i^N dy_i
\end{equation*}

Variation of $I$
\begin{equation*}
\begin{split}
	\delta I ={}& \int_{-\infty}^\infty \left[ \exp\left( -i (Y+\delta Y)^T S (Y+ \delta Y) + i D^T(Y + \delta Y) \right) - \exp\left( -i Y^T S Y + i D^T Y \right) \right]\prod_i^N dy_i \\
	={}& \int_{-\infty}^\infty \exp\left( -i Y^T S Y + i D^T Y \right) \left[\exp\left( -i(2Y^T S - D^T)\delta Y \right)-1\right] \prod_i^N dy_i  
\end{split}
\end{equation*}

To make $I$ stationary, $\delta I = 0$, that is
\begin{equation*}
	2\bar Y^T D - D^T = 0
\end{equation*}

Then we have
\begin{equation*}
	\bar Y = \frac12 S^{-1}D
\end{equation*}


At its stationary point, the integrand of $I$ becomes
\begin{equation*}
\begin{split}
	{}&\exp\left[ -i\left( \frac12 S^{-1}D \right)^T S \left(\frac12 S^{-1}D\right) + i D\left( \frac12 S^{-1}D \right) \right] \\
	={}& \exp\left(\frac14 i D^T S^{-1}D\right)
\end{split}
\end{equation*}
which is different from the integral result of $I$ with a prefactor $\sqrt{\pi^N/\det(iS)}$.







%%%%%%%  16.4  %%%%%%%%%%
\item

Denote
\begin{equation*}
	I = \int_{-\infty}^\infty \exp\left( -Y^T S Y + D^T Y \right)\prod_i^N dy_i 
\end{equation*}


Then
\begin{equation*}
\begin{split}
	\delta I ={}& \int_{-\infty}^\infty \left[ \exp\left(-(Y+\delta Y)^T S (Y+\delta Y) + D^T(Y+\delta Y)\right) -\exp\left( -Y^T S Y + D^T Y \right)  \right]\prod_i^N dy_i\\
	={}& \int_{-\infty}^\infty \exp\left( -Y^T S Y + D^T Y \right)\left[\exp\left( -(2Y^T S - D^T)\delta Y \right)-1\right]\prod_i^N dy_i
\end{split}
\end{equation*}

To make $I$ stationary, 
\begin{equation*}
	2\bar Y^T S - D^T =0 .
\end{equation*}
i.e.,
\begin{equation*}
	\bar Y = \frac12 S^{-1}D
\end{equation*}


Put $\bar Y = \frac12 S^{-1}D $ into the integrand of $I$, we get
\begin{equation*}
\begin{split}
	{}&\exp\left[ -\left(\frac12 S^{-1}D\right)^T S \frac12 S^{-1}D + D^T \frac12 S^{-1}D \right] \\
	={}& \exp\left( -\frac14 D^T S^{-1}SS^{-1}D  + \frac12 D^T S^{-1}D \right) \\
	={}& \exp\left( \frac14 D^T S^{-1}D \right)
\end{split}
\end{equation*}
which is the same as the integral result of $I$ apart from a prefactor.














%%%%%%%  16.5  %%%%%%%%%%
\item
\begin{eqnarray*}
	\langle q \vert e^{-itH} \rangle &=& \int\int  d p'  d p'' \langle q\vert p' \rangle \langle p'  \vert e^{-i p^2/(2m\hbar) t} \vert p''\rangle \langle p'' \vert 0 \rangle \\
	&=& \int  d p' \frac{1}{(2\hbar \pi)^3} e^{-\frac{-it}{2m\hbar} p'^2} e^{iqp/\hbar} \\
	&=& \frac{1}{(2\pi \hbar)^3} \sqrt{\frac{\pi^3}{(it/(2 m\hbar))}} e^{2mq^2/(2\hbar t)} \\
	&=& \left( \frac{m}{2\pi i\hbar t} \right)^{3/2} e^{imq^2/(2\hbar t)}
\end{eqnarray*}



%%%%%%%  16.6  %%%%%%%%%%
\item







%%%%%%%  16.7  %%%%%%%%%%
\item

\begin{eqnarray*}
	S[q] &=& \int_0^t \left( \frac 12 m\dot q^2 - \frac 1 2 m \omega^2  q^2 \right) d t' \\
	&=& \int_0^t \frac 1 2 m\left( (-\omega q' \sin\omega t' + \dot q_0 \cos\omega t')^2 -  \omega^2 (q' \cos\omega t' + \frac{\dot q^2}{\omega} \sin\omega t')^2 \right) d  t'\\
	&=& \frac{m\omega}{2\sin(\omega t)} \left( (q'^2 + q''^2)\cos\omega t - 2q'q'' \right)
\end{eqnarray*}







%%%%%%%  16.8  %%%%%%%%%%
\item

\begin{eqnarray*}
	S[\delta q] &=& \int_0^t d t' \left( \frac12 m \left( \sum_{n=1} a_n n\pi/t\cos\frac{n\pi t'}{t} \right)^2 -\frac12 m\omega^2\left( \sum_{n=1} a_n \sin\frac{n\pi t'}{t} \right)^2 \right) \\
	&=& \int_0^t  d t'\left( \frac12 m \sum_{n=1} a_n^2 \frac{n^2\pi^2}{t^2} \cos^2\frac{n\pi t'}{t} - \frac12 m\omega^2 \sum_{n=0} a_n^2 \sin^2\frac{n\pi t'}{t}  \right) \\
	&=& \sum_{n=1} \frac12 m a_n^2 \int_0^t  d t' \left( \frac{n^2 \pi^2}{t^2}\cos^2\frac{n\pi t'}{t} - \omega^2 \sin^2\frac{n\pi t'}{t}\right) \\
	&=& \sum_{n=1} \frac{mt}{4} a_n^2 \left( n^2 \pi^2/t^2 - \omega^2 \right)
\end{eqnarray*}






%%%%%%%  16.9  %%%%%%%%%%
\item


When $q'=0$ and $q''=q$, it becomes
\begin{eqnarray*}
	\langle q \vert e^{-i t H/\hbar } \vert 0 \rangle &=& \sqrt{\frac{m\omega}{2\pi i \hbar \sin\omega t}} \exp\left[ i\frac{m\omega[ q^2 \cos\omega t ]}{2\hbar \sin\omega t} \right]
\end{eqnarray*}

In the limit of $t\rightarrow 0$, the trigonometric functions used in our calculation becomes $\sin\omega t \rightarrow \omega t$ and $\cos\omega t \rightarrow 1$.

\begin{eqnarray*}
	\lim_{t\rightarrow 0}\langle q \vert e^{-i t H/\hbar } \vert 0 \rangle &=& \sqrt{\frac{m}{2\pi i \hbar t}} \exp\left( \frac{imq^2}{2\hbar t} \right)
\end{eqnarray*}





%%%%%%%  16.10  %%%%%%%%%%
\item





%%%%%%%  16.11  %%%%%%%%%%
\item

\begin{eqnarray*}
	S_e[q] &=& \int_0^{\beta} \left[ \frac12 m \dot q^2 +\frac12 m\omega^2 q^2 \right] d t \\
	&=& \int_0^{\beta} \frac12 m \left[ (A\omega e^{\omega t} - B \omega e^{-\omega t})^2 + \omega ( A e^{\omega t} + B e^{-\omega t})^2 \right] \\
	&=& \frac{1}{2} m\omega^2 \int_0^\beta 2( A^2 e^{2\omega t} + B^2 e^{-2 \omega t}  ) d t \\
	&=& m\omega^2 \left[ A^2 (e^{2\omega t} - 1) - B^2(e^{-2\omega t} - 1) \right]
\end{eqnarray*}



%%%%%%%  16.12  %%%%%%%%%%
\item

\begin{eqnarray*}
	S_0[\phi] &=& \int \frac12 [ -\partial_a \phi(x)\partial^a \phi(x) -m^2 \phi(x)]  d ^4x \\
	&=& \int \frac12 \left[ -\int ip_a e^{ip' x}\tilde \phi(p')\frac{1}{(2\pi)^4} d ^4p' \int\left(-ip_a e^{-ip'' x} \tilde \phi(-p'')\right) \frac{ d ^4p''}{(2\pi)^4}\right. \\
	&& \left. - m^2\int\int \frac{ d p'}{(2\pi)^4} \frac{ d p''}{(2\pi)^4} e^{ip'-p''}x\tilde \phi(p')\tilde\phi(-p'')  \right] d ^4 x \\
	&=& - \int  d ^4 x e^{i(p'-p'')x}\int\frac12 (p^2 + m^2)\tilde \phi (p')\tilde\phi(-p'')  \frac{ d ^4 p'}{(2\pi)^4}\frac{ d ^4p''}{(2\pi)^4} \\
	&=& - \delta(p'-p'')\int \frac12 (p^2 + m^2) \tilde\phi(p')\tilde\phi(-p'') \frac{ d ^4 p'}{(2\pi)^4}\frac{ d ^4p''}{(2\pi)^4} \\
	&=& -\frac12 \int\vert \tilde\phi(p)\vert^2 (p^2 + m^2) \frac{ d ^4 p}{(2\pi)^4}
\end{eqnarray*}



%%%%%%%  16.13  %%%%%%%%%%
\item

\begin{eqnarray*}
	&&\lim_{\epsilon\rightarrow 0+}\epsilon \int_{-\infty}^{\infty}e^{-\epsilon\vert t \vert}  dt \\
	&=&\lim_{\epsilon \rightarrow 0+} \left(  \epsilon \int_{-\infty}^0 f(t)e^{\epsilon t} dt + \epsilon \int_0^\infty f(t)e^{-\epsilon t} dt \right) \\
	&=& \lim_{\epsilon \rightarrow 0+} \left( \int_{-\infty}^0 f(t) de^{\epsilon t} - \int_0^\infty f(t) de^{-\epsilon t} \right) \\
	&=& \lim_{\epsilon \rightarrow 0+}  \left( f(t)e^{\epsilon t}\vert_{-\infty}^0 - \int_{-\infty}^0 e^{\epsilon t} df(t) - f(t)e^{-\epsilon t}\vert_0^{\infty} + \int_0^\infty e^{-\epsilon t}df(t)    \right) \\
	&=& \lim_{\epsilon\rightarrow 0} \left(  2f(0) + \int_0^\infty e^{-\epsilon t} d f(t) - \int_{-\infty}^0 e^{\epsilon t} d f(t)  \right) \\
	&=& 2f(0) + f(\infty) - f(0) - f(0) + f(-\infty) \\
	&=& f(\infty) + f(-\infty)
\end{eqnarray*}







%%%%%%%  16.14  %%%%%%%%%%
\item

{\color{red} Check this problem.}

Fourier transform of $\phi(\vec x,t)$ and $\phi(p)$ are
\begin{eqnarray*}
	\tilde \phi(\vec p,t) &=& \int e^{-i\vec p\cdot \vec x} \phi(\vec x,t) d^3 x
\end{eqnarray*}
\begin{eqnarray*}
	\phi(\vec x,t) &=& \int e^{i\vec p\cdot \vec x} e^{-ip_0 t}\phi(p')\frac{d^4 p'}{(2\pi)^4}
\end{eqnarray*}

Then
\begin{eqnarray*}
	\tilde \phi(\vec p,t)&=& \iint e^{-i\vec p\cdot \vec x} e^{-i \vec p'\cdot\vec x}e^{-i p_0 t} \phi(p') \frac{d^4}{(2\pi)^4} d^3 x \\
	&=& \iint e^{-i(\vec p - \vec p')\cdot \vec x}e^{-ip_0 t} \phi(p') \frac{d^4p'}{(2\pi)^4} d^3 x
\end{eqnarray*}





%%%%%%%  16.15  %%%%%%%%%%
\item

\begin{eqnarray*}
	S_0 [\phi,\epsilon,j]&=& -\frac12 \int \left[ \vert \tilde \phi(p) \vert^2 (p^2 + m^2 -i\epsilon) - \tilde j^*(p)\tilde\phi(p) - \tilde \phi^*(p)\tilde j(p) \right]\frac{d^4 p}{(2\pi)^4} \\
	&=& -\frac12 \int\left[ \left(\tilde \psi(p) + \frac{\tilde j(p)}{p^2 + m^2 - i\epsilon}\right)^* \left(\tilde \psi(p) + \frac{\tilde j(p)}{p^2 + m^2 - i\epsilon}\right) \left(p^2+m^2-i\epsilon\right) \right.\\
	&&\left. - \tilde j^*(p)\left(\tilde \psi(p) + \frac{\tilde j(p)}{p^2 + m^2 - i\epsilon}\right) - \left( \tilde\psi^*(p) + \frac{\tilde j^*(p)}{p^2+m^2+i\epsilon} \right)\tilde j(p) \right]\frac{d^4}{(2\pi)^4} \\
	&=& -\frac12 \int\left[ \vert \tilde \psi(p)\vert^2 (p^2 + m^2 - i\epsilon ) - \frac{\tilde j^*(p) \tilde j(p) }{ p^2 +m^2 -i\epsilon} \right] \frac{d^4p}{(2\pi)^4} \\
	&=& S_0[\psi,\epsilon] + \frac12 \int\frac{\tilde j^* (p) \tilde j(p)}{p^2 + m^2 - i\epsilon} \frac{d^4p}{(2\pi)^4}
\end{eqnarray*}



%%%%%%%  16.16  %%%%%%%%%%
\item

\begin{eqnarray*}
	Z_0[j] &=& \frac{ \int \exp\left[ i\int j(x)\phi(x)d^4x  \right] e^{iS_0[\phi,\epsilon] } D\phi }{ \int e^{i S_0[\phi,\epsilon]} D\phi } \\
	&=& \frac{\int e^{iS_0[\phi,\epsilon,j] }  D\phi  }{\int e^{i S_0[\phi,\epsilon]} D\phi} \\
	&=& \frac{ \int e^{i S_0[\psi,\epsilon]} D\psi \cdot e^{\frac i2 \int \frac{\tilde j^*(p) \tilde j(p)}{p^2+m^2-i\epsilon} \frac{d^4p}{(2\pi)^4}  }  }{ \int e^{i S_0[\psi,\epsilon]} D\psi}  \\
	&=& \exp\left[ \frac{i}{2} \int \frac{\vert\tilde j(p)\vert^2}{p^2 + m^2 -i\epsilon} \frac{d^4p}{(2\pi)^4} \right]
\end{eqnarray*}





%%%%%%%  16.17  %%%%%%%%%%
\item

Applying
\begin{eqnarray*}
	\tilde j(p) &=& \int e^{-ipx}j(x)d^4x \\
	\tilde j^*(p)&=& \int e^{ipx'} j(x')d^4x'
\end{eqnarray*}
to $Z_0[j]$, we get
\begin{eqnarray*}
	Z_0[j]&=& \exp\left[ \frac i2 \int \frac{\iint d^4x d^4 x' e^{ip(x-x')}j(x)j(x')}{ p^2 + m^2 -i\epsilon } \frac{d^4 p}{(2\pi)^4} \right] \\
	&=& \exp\left[ \frac i 2 \int j(x)\Delta (x-x')j(x') d^4x d^4x' \right],
\end{eqnarray*}
in which $\Delta(x-x')$ is the Feynmann's propagrator.






%%%%%%%  16.18  %%%%%%%%%%
\item

\begin{equation*}
\begin{split}
	&\left.\frac{1}{i^4}\frac{\delta^4 Z_0[j]}{\delta_j(x_1)\delta_j(x_2) \delta_j(x_3) \delta_j(x_4)}\right\vert_{j=0} \\
	={}& \left( Z_0[j] \int d^4x' d^4 x'' d^4 x''' d^4 x'''' \Delta(x_4 - x')\Delta(x_3 - x'')\Delta(x_2 - x''')\Delta(x_1 - x'''') \right.\\
	{}&j(x')j(x'')j(x''')j(x'''')\\
	{}& + \frac 1 i Z_0[j] \Delta(x_3 - x_4)\int d^4x' d^4 x'' \Delta(x_1 - x')\Delta(x_1 - x'')j(x')j(x'') \\
	{}& + \frac{1}{i} Z_0[j]\Delta(x_2 - x_4)\int d^4x' d^4 x'' \Delta(x_3 - x')\Delta(x_1 - x'')j(x')j(x'') \\
	{}& + \frac 1 i Z_0[j]\Delta(x_1 - x_4)\int d^4x' d^4 x'' \Delta(x_3 - x')\Delta(x_2 - x'')j(x')j(x'')\\
	{}& + \frac 1i Z_0[j]\Delta(x_2 - x_3)\int d^4x' d^4 x'' \Delta(x_3 - x')\Delta(x_2 - x'')j(x')j(x'')\\
	{}&+ \frac1i Z_0[j] \Delta(x_1 - x_3) \int d^4 x' d^4 x'' \Delta(x_2 - x') \Delta(x_4 - x'')j(x')j(x'') \\
	{}& + i^2 Z_0[j] \Delta(x_2 - x_3) \Delta(x_1-x_4) + i^2 Z_0[j] \Delta(x_1 - x_3)\Delta(x_2 - x_4) \\
	{}& + \frac1i Z_0[j]\Delta(x_1 - x_2)\int d^4x' d^4 x'' \Delta(x_3 - x')\Delta(x_4 - x'')j(x')j(x'')  \\
	{}& \left. \left. \frac{1}{i^2} Z_0[j]\Delta(x_3 - x_4)\Delta(x_1 - x_2)  \right)\right\vert_{j=0} \\
	={}& -\Delta(x_1 - x_4)\Delta(x_2 - x_3) - \Delta(x_2 - x_4)\Delta(x_1 -x_3) - \Delta(x_1 - x_2 )\Delta(x_3 - x_4)
\end{split}
\end{equation*}





%%%%%%%  16.19  %%%%%%%%%%
\item
\begin{equation*}
\begin{split}
	&\frac12 \int \left(\bm \nabla \Delta^{-1} j^0 \right)^2 d^4 x\\
	 ={}& \frac12 \int \left[ \bm \nabla \left( -\int \frac{1}{4\pi}\frac{j^0(x')}{\vert x-x'\vert} d^3x' \right) \right]^2 d^4x \\
	 ={}& \frac12 \iiint \frac{1}{4\pi} \frac{j(x')}{\vert x-x'\vert}\nabla^2 \frac{j(x'')}{\vert x'-x''\vert} d^3x' d^3 x'' d^4 x \\
	 ={}& \frac12 \iiint  \frac{1}{4\pi }\frac{j^)(x)j^0(x)}{\vert x-x'\vert} \delta(x'-x'')d^3 x d^3x''d^4x \\
	 ={}& \frac12 \iiint \frac{j^0(x)j^0(x')}{4\pi \vert x-x'\vert } d^3x' d^3x dt \\
	 ={}& \int V_c dt
\end{split}
\end{equation*}

That is
\begin{equation*}
	V_c=\frac12 \int \frac{j^0(\bm x,t)j^0(\bm y,t)}{4\pi \vert \bm x-\bm y\vert } d^3x d^3y
\end{equation*}




%%%%%%%  16.20  %%%%%%%%%%
\item


\begin{equation*}
\begin{split}
	S_0	 = {}& \int d^4x\left( -\frac14 F^{\mu\nu}F_{\mu\nu} + J^\mu A_\mu \right) \\
	={}&  \frac12 \int \frac{d^4k}{(2\pi)^4}\left[ -\tilde A_\mu(k)(k^2g^{\mu\nu} - k^\mu k^\nu)\tilde A_\nu(-k) + \tilde J^\mu(k)\tilde A_\mu(-k) + \tilde J^\mu (-k) \tilde A_\mu (k) \right]
\end{split}
\end{equation*}
in which $k^2 g^{\mu\nu - k^\mu k^\nu} = k^2 g^{\mu\nu}(k)$j.


Then we have
\begin{equation*}
\begin{split}
	Z_0={}&  \int DA e^{iS_0} \\
	={}& \exp\left[ \frac i 2 \int\frac{d^4k}{(2\pi)^4}\tilde J_\mu (k) \frac{j^{\mu\nu}(k)}{k^2 - i \epsilon}\tilde J_\nu (-k) \right]
\end{split}
\end{equation*}

Finally
\begin{equation*}
\begin{split}
	{}&\langle 0 \vert \mathcal T[A_\mu(x)A_\nu(y)]\vert 0\rangle\\
	={}& \int\frac{y_{\mu\nu}}{k^2 - i\epsilon} e^{ik(x-y)}\frac{d^4 k}{(2\pi)^4}
\end{split}
\end{equation*}



%%%%%%%  16.21  %%%%%%%%%%
\item

Using
\begin{equation*}
\begin{split}
	\vert \theta \rangle = {}& \exp\left( \psi^\dag \theta - \frac12 \theta^* \theta  \right) \vert 0 \rangle \\
	={}& \left( 1+\psi^\dag\theta - \frac12 \theta^*\theta \right)\vert 0\rangle
\end{split}
\end{equation*}
and
\begin{equation*}
\begin{split}
	\langle \xi \vert ={}& \langle 0\vert \left( 1+\xi^* \psi - \frac12 \xi^*\xi \right),
\end{split}
\end{equation*}
we get
\begin{equation*}
\begin{split}
	\langle \xi \vert \theta\rangle ={}&  \langle 0 \vert \left( 1 + \xi^* \psi - \frac12 \xi^*\xi \right)\left( 1 + \psi^\dag \theta - \frac12 \theta^* \theta \right) \vert 0 \rangle \\
	={}& \langle 0 \vert 1 + \xi^* \psi \psi^\dag \theta  - \frac12 \xi^* \xi - \frac12 \theta^* \theta  + \frac14 \xi^*\xi\theta^*\theta \vert 0 \rangle \\
	={}& \langle 0 \vert 1+\xi^* \theta  - \frac12 \xi^* \xi - \frac12 \theta^* \theta + \frac14 \xi^* \xi\theta^* \theta\vert 0 \rangle \\
	={}& \exp\left( \xi^*\theta - \frac12 (\xi^*\xi + \theta^* \theta) \right)
\end{split}
\end{equation*}



%%%%%%%  16.22  %%%%%%%%%%
\item

\begin{equation*}
\begin{split}
	{}&\int \vert \theta \rangle\langle \theta \vert d\theta ^* d \theta \\
	={}&\left( 1+\psi^\dag \theta - \frac12 \theta^*\theta  \right) \vert 0 \rangle \langle 0\vert \left( 1 + \theta^* \psi - \frac12 \theta^* \theta \right)d\theta^* d\theta \\
	={}& \int \left( \vert 0\rangle \langle 0 \vert \theta \vert 1 \rangle \langle 0\vert - \frac12 \theta^* \theta \vert 0 \rangle \langle 0 \vert + \theta\theta^* \vert 1 \rangle \langle 1 \vert   - \frac12 \theta^* \theta \vert 0 \rangle \langle 0 \vert  \right) d\theta^* d\theta \\
	={}& \vert 0 \rangle \langle 0 \vert  + \vert 1 \rangle \langle 1 \vert \\
	={}& I
\end{split}
\end{equation*}





%%%%%%%  16.23  %%%%%%%%%%
\item

We can expand the state
\begin{equation*}
\begin{split}
	\vert \theta \rangle ={}& \exp\left( \sum_{k=1}^n \psi_k^\dag \theta_k - \frac12 \theta_k^* \theta_k \right) \vert 0 \rangle \\
	={}& \left[ \prod_{k=1}^n \left( 1 + \psi_k^\dag \theta_k - \frac12 \theta_k^* \theta_k \right) \right]\vert 0 \rangle
\end{split}
\end{equation*}

The ground state is
\begin{equation*}
	\vert 0 \rangle = \left( \prod_{k=1}^n \psi_k \right)\vert s\rangle
\end{equation*}

Then
\begin{equation*}
\begin{split}
	\psi_k\vert \theta \rangle = {}& \prod_{i\neq k }^n \left( 1 + \psi_i^\dag \theta_i - \frac12 \theta_i ^* \theta_i \right) \psi_k \left( 1 + \psi_k^\dag \theta_k - \frac12 \theta_k^* \theta_k \right)\vert 0 \rangle \\
	={}& \prod_{i \neq k}^n \left( 1 + \psi_i^\dag \theta_i - \frac12 \theta_i^*\theta_i  \right)\psi_k\psi_k^\dag \theta_k \vert 0 \rangle \\
	={}& \prod_{i\neq k}^n \left( 1 + \psi_i^\dag \theta_i - \frac12 \theta_i^* \theta_i \right)\left( 1 - \psi_k^\dag\psi_k \right)\theta_k \vert 0 \rangle \\
	={}& \theta_k \prod_{k=1}^n \left( 1 + \psi_k^\dag \theta_k - \frac12 \theta_k^*\theta_k \right) \vert 0 \rangle \\
	={}&\theta_k \vert \theta \rangle
\end{split}
\end{equation*}


%\end{comment}
%%%%%%%  16.24  %%%%%%%%%%
\item

\begin{equation*}
	\vert 0 \rangle = \left( \prod_{m} \psi_m(\bm x,0) \right)\vert s\rangle
\end{equation*}

\begin{equation*}
\begin{split}
	\vert \chi \rangle ={}& \exp\left[ \int \sum_m \psi_m^\dag(\bm x,0)\chi_m(\bm x) - \frac12 \chi_m^* (\bm x) \chi_m(\bm x)d^3 x \right]\vert 0 \rangle \\
	={}& \exp\left[ \int\left( \psi^\dag \chi - \frac12 \chi^\dag \chi \right)d^3x \right] \vert 0 \rangle
\end{split}
\end{equation*}


\begin{equation*}
	\psi_m(\bm x,0) \vert \chi \rangle = \exp\left[ \int \sum_{i \neq m} \left( \psi_i^\dag (\bm x,0)\chi_m(\bm x) - \frac12 \chi_m^*(\bm x) \chi_m(\bm x) \right) d^3 x  \right]\vert 0 \rangle
\end{equation*}



\begin{equation*}
\begin{split}
	{}&\psi_m(\bm x,0) \left [\int  \left( 1 + \psi_m^\dag(\bm x, 0 )\chi_m(\bm x) - \frac12 \chi_m^* (\bm x)\chi_m(\bm x)  \right) d^3x \right]  \vert 0   \rangle \\
	={}& \exp\left[ \int  \sum_{i\neq m} \left( \psi_i^\dag(\bm x,0) \chi_i(\bm x) - \frac12 \chi_i^*(\bm x)\chi_i(\bm x)  \right)d^3x  \right] 
\end{split}
\end{equation*}



\begin{equation*}
\begin{split}
	{}&( 1 - \psi_m \psi_m^\dag) \chi_m(\bm x) \vert 0 \rangle \\
	={}& \chi_m(\bm x) \exp\left[ \int \sum_m \left( \psi_m^\dag(\bm x,0) \chi_m(\bm x) - \frac12 \chi_m^*(\bm x)\chi_m \right) d^3x \right] \vert 0 \rangle \\
	={}& \chi_m(\bm x)\vert \chi\rangle
\end{split}
\end{equation*}





%%%%%%%  16.24  %%%%%%%%%%
\item

\begin{equation*}
	\vert \chi\rangle = \exp\left[ \int \left( \psi^\dag \chi - \frac12 \chi^\dag \chi \right)d^3x \right]\vert 0 \rangle 
\end{equation*}

\begin{equation*}
	\langle \chi \vert  = \langle 0 \vert \left[ \exp\int\left( \chi^\dag \psi - \frac12 \chi^\dag \chi \right)d^3 x \right]
\end{equation*}

\begin{equation*}
\begin{split}
	\langle \chi \vert \chi \rangle  ={}& \langle 0 \vert \exp \left(\int (\psi^\dag \chi+ \chi'^\dag \psi - 1/2 \chi^\dag \chi - 1/2 \chi'\dag \chi' )d^3x \right)\vert 0 \rangle \\
	={}& \langle 0 \vert  \int \prod_m \left( 1 + \psi_m^\dag \chi_m + \chi_m'^\dag \psi_m - \frac12 \chi_m^\dag \chi_m - \frac12 \chi_m'^\dag \chi_m' + \psi_m^\dag \chi_m \chi_m'^\dag \psi_m \right)d^3 x  \vert  0 \rangle \\
	={}& \langle 0 \vert \int\prod_m \left( 1 + \chi_m'^\dag \chi_m - \frac12 \chi_m^\dag \chi_m   - \frac12 \chi_m'^\dag \chi_m'  \right)d^3x  \vert 0 \rangle \\
	={}& \exp\left[ \int\left( \chi'^\dag \chi - \frac12 \chi'^\dag \chi' - \frac12 \chi^\dag \chi  \right)d^3 x \right]
\end{split}
\end{equation*}
































\end{enumerate}











\end{document}